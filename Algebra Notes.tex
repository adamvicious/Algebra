\documentclass[11pt, oneside]{article}   	% use "amsart" instead of "article" for AMSLaTeX format
\usepackage{geometry}                		% See geometry.pdf to learn the layout options. There are lots.
\geometry{letterpaper}                   		% ... or a4paper or a5paper or ... 
%\geometry{landscape}                		% Activate for for rotated page geometry
%\usepackage[parfill]{parskip}    		% Activate to begin paragraphs with an empty line rather than an indent
\usepackage{graphicx}				% Use pdf, png, jpg, or eps§ with pdflatex; use eps in DVI mode
								% TeX will automatically convert eps --> pdf in pdflatex		
\usepackage{amssymb}

\title{Adam's Abstract Algebra Almanac}
\author{Adam Howard}
\date{\today}							% Activate to display a given date or no date

\begin{document}
\maketitle
\section*{Introduction}
This collection of notes is in preparation for my comprehensive exam in Algebra. They will contain the basic definitions, many basic propositions with various levels of proof (none, sketch, complete, full-blown unnecessary analysis) and my solutions to exercises I find interesting or relevant. The references include but are not limited to: Paola Aluffi's \textit{Algebra: Chapter 0}, Michael Artin's \textit{Algebra}, Nathan Jacobson's \textit{Basic Algebra 1/2}, Emily Rehiel's \textit{Category Theory in Context}, and the Ether (otherwise known as the internet.)
\section*{Groups}
\section*{Rings}
\text


\section*{Examples}

\section*{Problems}
\subsection{}
1. Show that any finite group of even order contains an element of order 2. \newline
2. Is the additive group of integers isomorphic to the additive group of rationals? \newline
3. Show that any finite generated subgroup of the additive group of rationals is cyclic. Use this to show that the additive group of rationals is not $\mathbb{Q} \oplus \mathbb{Q}$ where the group operation is vector addition.  \newline
4. Write $$(456)(567)(671)(123)(234)(345)$$ as a product of disjoint cycles. \newline
5. Show that if $n \geq 3$ that $A_{n}$ is generated by 3-cycles.




\end{document}  